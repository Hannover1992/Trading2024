\subsection{Reward-Funktion für Trading-Bots}

\paragraph{Kompakte Formel:}
\[
\text{Reward}_t = \frac{\text{Gain}_t - \text{BestScenarioGain}_t}{V_t}
\]

wobei:
\[
\text{Gain}_t = V_t - V_{t-1}
\]

\[
\text{BestScenarioGain}_t = \max\left(0, \text{PotentialGainIfFullyInvested}_t - V_{t-1}\right)
\]

\[
\text{PotentialSharesIfFullyInvested}_t = \frac{C_{t-1} \times \text{TransactionPenalty}}{P_{t-1}} + S_{t-1}
\]

\[
\text{PotentialGainIfFullyInvested}_t = \text{PotentialSharesIfFullyInvested}_t \times P_t \times \text{TransactionPenalty}
\]

\paragraph{Erläuterung:}

1. \textbf{Realer Gewinn} (\(\text{Gain}_t\)): Die Differenz zwischen dem aktuellen Portfoliowert (\(V_t\)) und dem vorherigen Portfoliowert (\(V_{t-1}\)).

2. \textbf{Bestes Szenario} (\(\text{BestScenarioGain}_t\)): Der Gewinn im besten hypothetischen Szenario, falls der Preis steigt, abzüglich des vorherigen Portfoliowerts (\(V_{t-1}\)). Falls der Preis fällt, ist der Gewinn im besten Szenario 0.

   - \textbf{Hypothetischer Aktienbestand} (\(\text{PotentialSharesIfFullyInvested}_t\)): Die Anzahl der Aktien, die man hätte kaufen können, wenn man im vorherigen Zeitschritt alle verfügbaren Barmittel in Aktien investiert hätte, plus die bereits vorhandenen Aktien (\(S_{t-1}\)).
     \[
     \text{PotentialSharesIfFullyInvested}_t = \frac{C_{t-1} \times \text{TransactionPenalty}}{P_{t-1}} + S_{t-1}
     \]

   - \textbf{Hypothetischer Gewinn} (\(\text{PotentialGainIfFullyInvested}_t\)): Der hypothetische Gewinn, der sich aus dem hypothetischen Aktienbestand, dem aktuellen Aktienpreis und den Transaktionskosten ergibt.
     \[
     \text{PotentialGainIfFullyInvested}_t = \text{PotentialSharesIfFullyInvested}_t \times P_t \times \text{TransactionPenalty}
     \]

3. \textbf{Normalisierung} (\(\text{Reward}_t\)): Die Differenz zwischen dem realen Gewinn und dem Gewinn im besten Szenario, geteilt durch den aktuellen Portfoliowert. Dadurch wird die Belohnung in Relation zum aktuellen Portfoliowert gesetzt.

\paragraph{Interpretation:}

- \textbf{Positive Belohnung:} Der Agent hat besser abgeschnitten als im besten hypothetischen Szenario (oder zumindest gleich gut, wenn der Preis gefallen ist).
- \textbf{Negative Belohnung:} Der Agent hat schlechter abgeschnitten als im besten hypothetischen Szenario (d.h., er hätte mehr Gewinn machen können, wenn er anders gehandelt hätte).
- \textbf{Null Belohnung:} Der Agent hat genauso gut abgeschnitten wie im besten hypothetischen Szenario.

\paragraph{Vorteile der kompakten Formel:}

- \textbf{Übersichtlicher:} Die Formel ist kürzer und leichter zu erfassen.
- \textbf{Intuitiver:} Die einzelnen Komponenten sind klarer benannt und ihre Bedeutung ist leichter verständlich.
