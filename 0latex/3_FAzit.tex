\chapter{Diskussion}
\label{sec:Discussion}

In dieser Studie wurde der Deep Deterministic Policy Gradient (DDPG) Algorithmus zur Optimierung von PID-Reglern in DC-DC-Wandlersystemen erfolgreich angewendet. Die Ergebnisse heben die Effektivität des DDPG-Ansatzes hervor, insbesondere im Hinblick auf die Sensibilität bezüglich der Hyperparameterwahl und der Gestaltung der Belohnungsfunktion. Die systematische Untersuchung und das Training des DDPG-Algorithmus über mehrere Phasen haben zu einem tiefgreifenden Verständnis der Regelungsdynamik eines PID-regulierten DC-DC-Konverters geführt. Die schrittweise Entwicklung vom ersten Erlernen der Umgebungsbedingungen über die Fehlleitung durch einen untrainierten Kritiker bis hin zur Feinabstimmung und nahezu optimalen Regelungsleistung hat mehrere Schlüsselelemente des maschinellen Lernens hervorgehoben.

\paragraph{Vergleich mit Bayesianischer Optimierung}
Ein interessanter Aspekt der Untersuchung war der Vergleich des DDPG-Ansatzes mit der Bayesianischen Optimierung. Obwohl es keine direkte Baseline-Methode gab, ermöglichte dieser Vergleich eine Einschätzung der Effektivität beider Ansätze. Beide Verfahren führten zu ähnlichen Ergebnissen, was darauf hindeutet, dass der DDPG-Algorithmus eine gleichwertige, wenn nicht sogar überlegene Alternative in bestimmten Szenarien sein kann.

\paragraph{Phasen der Untersuchung}
Im Verlauf der Studie wurden verschiedene Phasen betrachtet, von der initialen Anwendung des DDPG-Algorithmus über die Analyse verschiedener Netzwerkgrößen bis hin zur Miniaturisierung der Modelle. Die Robustheit des DDPG-Algorithmus wurde dabei deutlich, insbesondere seine Fähigkeit, auch bei reduzierten Modellgrößen effektiv zu konvergieren.

\paragraph{Miniaturisierung und Leistung}
Die Ergebnisse aus der Phase der Miniaturisierung sind besonders bemerkenswert. Trotz der signifikanten Reduktion in der Größe des Netzwerks blieb die Leistung der Modelle hoch. Dies unterstreicht das Potenzial des DDPG-Ansatzes für Anwendungen, bei denen eine kompakte und effiziente Modellarchitektur erforderlich ist.

