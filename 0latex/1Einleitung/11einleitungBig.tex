\chapter{Einleitung}
\section{Hintergrund und Motivation}
Die effiziente Übertragung elektrischer Energie von einer Quelle zu einem Verbraucher stellt eine entscheidende Herausforderung in der Elektrotechnik dar. Im Mittelpunkt stehen dabei Gleichspannungswandler (DC-DC Konverter), die in vielfältigen Anwendungen von Energieübertragungssystemen bis hin zu mobilen Geräten eine Rolle spielen \cite[p.~70]{wensdesign2022}.

\section{Problemstellung}
Jedoch werden diese Systeme durch Degradationseffekte, besonders von Schlüsselkomponenten wie Kapazitäten, zunehmend beeinträchtigt. Solche Degradationen können die Lebensdauer und Effizienz von elektronischen Systemen nachhaltig schädigen und erfordern dringende wissenschaftliche Untersuchung \cite[p.~1]{jeong2023degradation}.

\section{Relevanz und Forschungslage}
Erste Studien in diesem Bereich, wie die von Jeong et al. und Kulkarni et al., haben bereits die drastischen systemischen Auswirkungen solcher Degradationen demonstriert \cite[p.~3]{kulkarni_model-based_2023}. Dadurch wird die Notwendigkeit unterstrichen, innovative Lösungsansätze für dieses Problem zu entwickeln.

\section{Forschungsmethode und Ansatz}
Ein vielversprechender Lösungsansatz ist die Nutzung von künstlichen neuronalen Netzen (KNN). Diese bieten, wie die Arbeiten von Brunton und Kutz sowie Almawlawe et al. nahelegen, eine hervorragende Plattform für die Steuerung komplexer Systeme und könnten somit eine Alternative zu traditionellen Reglern bieten \cite[p.~270]{brunton2019data} \cite[p.~8]{Almawlawe2023}.

\section{Ziele und Aufbau der Arbeit}
Diese Arbeit zielt darauf ab, die Anwendungsmöglichkeiten von KNN zur Überwachung und Kompensation der Degradation in DC-Konvertern systematisch zu untersuchen. Im Fokus stehen dabei die Architektur des neuronalen Netzes, verschiedene Trainingsmethoden und -umgebungen, sowie die spezifischen Herausforderungen und Lösungsansätze im Kontext des Trainingsprozesses.
